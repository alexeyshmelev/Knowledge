\documentclass[a4paper,14pt]{article}
\newcommand{\comment}[1]{}
\usepackage{cmap}
\usepackage{mathtext}
\usepackage[T2A]{fontenc}
\usepackage[utf8]{inputenc}
\usepackage[english,russian]{babel}
\usepackage{soulutf8}
\usepackage{babel}
\usepackage{amsmath,amsfonts,amssymb,amsthm,mathtools}
\usepackage{icomma}
\usepackage{enumitem}
\usepackage{extsizes}
\usepackage{indentfirst}
\usepackage[font={small}]{caption}
\usepackage{fancyhdr}
\pagestyle{fancy}
\renewcommand{\headrulewidth}{1mm}
\lhead{GRENLEX}
\rhead{Physics}
\usepackage{color}
\definecolor{orange}{RGB}{255, 136, 0}
\usepackage{hyperref}
\usepackage[usenames,dvipsnames,svgnames,table,rgb]{xcolor}
\hypersetup{				
	unicode=true,           
	pdftitle={Заголовок},   
	pdfauthor={Автор},      
	pdfsubject={Тема},      
	pdfcreator={Создатель}, 
	pdfproducer={Производитель}, 
	pdfkeywords={keyword1} {key2} {key3},
	colorlinks=true,       	
	linkcolor=black,       
	citecolor=black,     
	filecolor=magenta,  
	urlcolor=orange      
}
\usepackage{tikz}
\usepackage{float}

\newcommand{\mysubsection}[1]{\subsection*{#1}
\addcontentsline{toc}{subsection}{#1}} 

\newcommand{\mysection}[1]{\section*{#1}
\addcontentsline{toc}{section}{#1}} 

%--------------------------------------------------------------------------------

\begin{document}
	\thispagestyle{empty}
	%\begin{center}
	%	\textbf{GRENLEX STUDY}
	%\end{center}

\vspace{10ex}
	
		\begin{center}
		
		\textbf{Физика} \\
		\textit{Шмелев Алексей Валерьевич}
		
	\end{center}

\vfill
	\begin{center}
	\small{GRENLEX STUDY 2020}
	\end{center}



	\newpage
	
	\tableofcontents
	
	
	
	\newpage
	
\mysection{Квантовая оптика, атомная и ядерная физика}
\mysubsection{Первая модельатома (модель  Томсона)  и ее недостатки}

Атом, по Томсону, состоит из электронов, помещённых в положительно заряженный «суп», компенсирующий электрически отрицательные заряды электронов, образно — подобно отрицательно заряженным «изюминкам» в положительно заряженном «пудинге». Электроны, как предполагалось, были распределены по объёму всего атома.

Недостатки модели Томсона:

\begin{enumerate}
	\item Не объяснила дискретный характер излучения атома и его устойчивость (она  не была в состоянии описать адекватно спектральные закономерности излучения  атома)
	\item Не даёт возможности понять, что определяет размеры атомов
	\item Оказалась в полном противоречии с опытами по исследованию распределения положительного заряда в атоме (опыт Резерфорда)
\end{enumerate}

\mysubsection{Опыт Резерфорда. Ядерная  (планетарная)  модель  атома  и  ее недостатки}

Резерфорд предложил зондировать вещество альфа-частицами для того, чтобы найти экспериментальный  метод,  который,  отличаясь  от  спектральных  методов, ответил бы прямо на вопрос об истинной структуре атома. Задача,  поставленная  Резерфордом,  состояла  в  том,  чтобы  по  угловому распределению   рассеянных   альфа-частиц   ответить   на   вопрос,   как распределены масса и заряд атома.

Недостатки модели Резерфорда:

\begin{enumerate}
	\item Эта модель не согласуется с наблюдаемой стабильностью атомов. По законам классической электродинамики, вращающийся вокруг ядра электрон должен \textbf{непрерывно} излучать электромагнитные волны, а поэтому терять свою энергию. В результатеи электроны будут постепенно приближаться к ядру и, в конечном итоге, упадут на него.
	\item Эта модель не объясняет наблюдаемые в опытах оптические спектры атомов. Как следут из теории Резерфорда, оптические спектры атомо не непрерывны, а состоят из узких спектральных линий, т.е. атомы излучают и поглощают электромагнитные волны лишь определённых частот, характерных для данного химического элемента.
\end{enumerate}

\mysubsection{Постулаты Бора. Теория атома водорода по Бору.}

\begin{enumerate}
	\item Атомная система может находиться только в особых стационарных состояниях, каждому из которых соответствует определённая энергия $E_n$. В стационарном состоянии атом не излучает.
	\item При переходе атома из одного стационарного сотояния в другое испускается или поглощается квант энергии $E=h\nu$.
	\item Электроны могут двигаться по орбитам определённого радиуса. На этой орбите момент импульса электрона равен приведённой постоянной Планка:
	\[mvr=k\hbar\]
\end{enumerate}

Найдём формулу радиуса стационарных орбит. Для этого составим систему:

\begin{equation*}
\begin{cases}
\dfrac{mv^2}{r} = \dfrac{e^2}{r^2} \\
mvr=n\hbar
\end{cases}
\end{equation*}

Из этой системы мы получаем, что:

\[r_n=\dfrac{\hbar^2}{me^2}n^2\]

Получим формулы для кинетической, потенциальной и полной энергии электрона на стационарных орбитах:

Полная энергия электрона складывается из его кинетической и понетциальной энергии:

\[E=E_{\text{к}}+E_{\text{п}}\]

Найдём $E_{\text{п}}$:

При $r \rightarrow \infty, E_{\text{п}} \rightarrow 0$, поэтому:

\[E_{\text{п}} = \int\limits_\infty^r F dr = \int\limits_\infty^r \dfrac{e^2}{4\pi \varepsilon_0r^2}dr = - \dfrac{e^2}{4\pi\varepsilon_0r}\]

Найдём $E_{\text{к}}$:

\[E_{\text{к}} = \dfrac{mv^2}{2}\]

Но мы не знаем скорость электрона. Её можно получить из второго закона Ньютона:

\[\dfrac{mv^2}{r} = \dfrac{e^2}{4\pi\varepsilon_0r^2}\]

Отсюда:

\[E_{\text{к}} = \dfrac{mv^2}{2} = \dfrac{e^2}{8\pi\varepsilon_0r}\]

Обобщённая формула Бальмера:

\[\nu = R \left(\dfrac{1}{m^2} - \dfrac{1}{n^2}\right)\]

Постоянная Ридберга ($R'$) получается из имперической формулы, описывающей все известные в то время (19 век) спектральные линии атома водорода в \textit{видимой области спектра}:

\[\dfrac{1}{\lambda} = R' \left(\dfrac{1}{m^2} - \dfrac{1}{n^2}\right)\]

Причём $R=R'c$ --- тоже постоянная Ридберга.


\mysubsection{Опыты Франка и Герца}

Опыт, являющийся экспериментальным доказательством дискретности внутренней энергии атома.

\mysubsection{Корпускулярно-волновой  дуализм  свойств  вещества. Гипотеза  де Бройля}

Гипотеза де Бройля: с каждым микрообъектом связывается, с одной стороны, корпускулярные характеристики --- энергия $E$ и импульс $p$, а с другой --- волновые характеристики --- $\nu$ и длина волны $\lambda$. Он утверждал, что не только фотоны, но и электроны и любые другие частицы материи наряду с корпускулярными обладают также и волновыми свойствами.

\mysubsection{Уравнение  Шредингера.  Уравнение  Шредингера  для  стационарных состояний}

Уравнение Шрёдингера имеет вид:

\[-\dfrac{\hbar^2}{2m}\Delta\Psi + U(x, y, z, t)\Psi = i\hbar\dfrac{\partial \Psi}{\partial t}\]

где $\hbar = \dfrac{h}{2\pi}$, $m$ --- масса частицы, $\Delta$ --- оператор Лапласа $(\Delta\Psi = \dfrac{\partial^2\Psi}{\partial x^2} + \dfrac{\partial^2\Psi}{\partial y^2} + \dfrac{\partial^2\Psi}{\partial z^2})$, $i$ --- мнимая единица, $U(x, y, z, t)$ --- потенциальная функция частицы в силовоим поле, в котором она движется, $\Psi(x, y, z, t)$ --- искомая волновая функция частицы.

Для многих физических явлений, происходящих в микромире, общее уравнение Шрёдингера можно упростить, исключив зависимость $\Psi$ от времени.

\[\Delta\psi + \dfrac{2m}{\hbar^2}(E-U)\psi = 0\]

Это уравнение назвается уравнением Шрёдингера для стационарных состояний.

\mysubsection{Уравнение Шредингера для атома водорода. Квантовые числа}

Применить стационарное уравнение Шредингера к атому водорода это значит:

\begin{enumerate}
	\item Подставить в это уравнение выражение для потенциальной энергии взаимодействия электрона с ядром:
	
	\[U = -\dfrac{1}{4\pi\varepsilon_0} \cdot \dfrac{e^2}{r}\]
	
	\item В качестве $m$ подставить $m_e$ - массу электрона
\end{enumerate}

После этого получим уравнение Шредингера для атома водорода:

\[-\dfrac{\hbar^2}{2m_e}\Delta\Psi -\dfrac{1}{4\pi\varepsilon_0} \cdot \dfrac{e^2}{r}\Psi = i\hbar\dfrac{\partial \Psi}{\partial t}\]

Для описания положения и энергии электрона в атоме используется четыре квантовых числа. Эти числа можно рассматривать как некие коэффициенты в решениях важнейшего в квантовой механике уравнения Шредингера. 

\begin{enumerate}
	\item Главное квантовое число n эквивалентно квантовому числу в теории Бора. Оно в основном определяет энергию электронов на данной орбитали.
	\item Орбитальное квантовое число l определяет значение орбитального момента количества движения электрона на данной орбитали (характеризует форму орбитали).
	\item Магнитное квантовое число $m_l$ определяет значение составляющей проекции момента количества движения электрона на выделенное направление в пространстве.
	\item Спиновое квантовое число, характеризует веретенообразное движение электрона вокруг своей оси.
\end{enumerate}










\mysection{Квантовая природа излучения}
\mysubsection{Тепловое излучение}

Это свечение тел, обусловленное нагреванием.

Особенности: 

\begin{itemize}
	\item Является равновесным (т.е. тело в единицу времени поглощает столько же энергии, сколько и излучает)
\end{itemize}

\mysubsection{Спектральная плотность энергетической светимости}

Это мощность излучения с единицы площади поверхности тела в интервале частот единицной длины, т.е.
\[R_{\nu,T}=\dfrac{dW^{\text{изл}}_{\nu,\nu+d\nu}}{d\nu}\]

где $dW^{изл}_{\nu,\nu+d\nu}$ --- энергия электромагнитного излучения, испускаемого за единицу времени (мощность злучения) с единицы площади поверхности тела в интервале частот от $\nu$ до $\nu+d\nu$.

\mysubsection{Спектральная поглощательная способность}

Это способность тел поглощать падающее на них излучение. Задаётся формулой:
\[A_{\nu,T}=\dfrac{dW^{\text{погл}}_{\nu,\nu+d\nu}}{dW_{\nu,\nu+d\nu}}\]

\mysubsection{Чёрное тело}

Это тело, способное полностью поглощать всё падающее на него излучение любой частоты при  любой температуре. Следовательно, спектральная поглощательная способность чёрного тела для всех частот и температур равна единице. Чёрных тел в природе нет.

\mysubsection{Закон Кирхгофа}

Отношение спектральной плотности энергетической светимости к спектральной поглощательной способности не зависит от природы тела; оно является для всех тел универсальной функцией частоты и температуры:

\[\dfrac{R_{\nu,T}}{A_{\nu,T}}=r_{\nu,T}\]

\mysubsection{Закон Стефана---Больцмана}

\[R_e=\sigma T^4\]

\mysubsection{Закон смещения Вина}

\[\lambda_{max}=\dfrac{b}{T}\]

\mysubsection{Эксперименты, подтверждающие квантовые свойства света}

\begin{itemize}
	\item \textbf{Опыт А.Ф.Иоффе и Н.И.Добронравова.}
	
	В электрическом поле плоского конденсатора уравновешивалась заряженная пылинка из висмуса. Нижняя обкладка конденсатора изготавливалась тончайшей алюминиевой фольги, которая являлась одновременно анодом миниатюрной рентгеновской трубки. Анод бомбардировался ускоренными до 12кВ фотоэлектронами, испускаемыми катодом под действием ультрафиолетового излучения. Освещённость катода подбиралась столь слабой, чтобы из него в 1 секунду вырывалось лишь 1000 фотэлектронов, а следовательно, и число рентгеновских импульсов было 1000 в 1 секунду. Опыт показал, что в среднем через каждые 30 мин уравновешенная пылинка выходила из равновыесия, т.е. рентгеновское излучение освобождало из неё фотоэлектрон.
	
	\item \textbf{Опыт С.И.Вавилова}
	
	Наблюдал флуктуации слабых потоков \textit{видимого света}. Наблюдения проводились визуально. Глаз, адаптированный к темноте, обладает довольно резким порогом зрительного ощущения, т.е. воспринимает свет, интенсивность которого не меньше некоторого порога. С.И.Вавилов наблюдал периодически повторяющиеся вспышки света одинаковой длительности. С уменьшением светового потока некоторые вспышки уже не воспринимались глазом, причём чем слабее был световой поток, тем больше было пропусков вспышек. Это объясняется флуктуациями интенсивности света, т.е. число фотонов оказывалось по случайным причинам меньше порогового значения. \textit{Таким образом, опыт Вавмлова явимлся наглядным подтверждением квантовых свойств света.}
	
\end{itemize}

\mysubsection{Экспериментальное подтверждение гипотизы де Бройля}
 Дэвиссон и Джермер в 1927 году исследовали отражение электронов от монокристала никеля, имеющего кубическую кристаллическую решётку. Интенсивность отражения они оценивали по силе тока гальванометра.
 
 Оценим длину волны де Бройля, отвечающую энергии электрона 54 эВ:
 \[\lambda=\dfrac{h}{p}=\left[T=\dfrac{p^2}{2m} \Rightarrow p=\sqrt{2mT}\right]=\dfrac{h}{2mT}=1.67 \cdot 10^{-10}\]

Если наблюдается дифракция, то должно выполняться условие Вульфа-Брэгга, где $d_{Ni}$ --- период кристаллической решётки никеля:
\[2d\sin \theta = n\lambda, d_{Ni} = 0.91 \cdot 10^{-10} \text{м}\]

Брэгговская длина волны и длина волны де Бройля практически совпадают.

Ещё один эксперимент, подтверждающий гипотезу де Бройля. Напомним, гипотеза де Бройля говорит от том, что каждой движущейся частице можно сопоставить некоторый волновой процесс. Томсон (1927 года) и независимо от него Тартаковский получили дифракционную картину при прохождении электронного пучка через металлическую фольгу. Возникает вопрос: волновыми свойствами обладает именно пучок электронов (т.е. когда их много) или отдельно каджый электрон? де Бройль полагал, что \textbf{каждой} частице можно сопроставить некоторый волновой процесс. Но остаётся вопрос, что же представляет из себя этот волновой процесс? Дело в том, что электрон в экспериментах всегда представляет себя как единое целое. Т.е. никто никогда не наблюдал пол электрона или его часть.

Существуют два предположения о том, что никакого дуализма волн и частиц нет:
\begin{enumerate}
\item Существуют только волны. Частица представляет собой суперпозицию волн.
\item Первичными являеся частицы, а волны возникают в среде, состоящей из частиц (точно какже, как возникает олна на поверхности воды).
\end{enumerate}

Из волн различных частот всегда можно составить волновой пакет, т.е. такое волновое образование, что при наложении волны будут усиливать друг друга в какой-то малой области простарнства, а вне этой области произойдёт их полное гашение.

Фазовая скорость волн де Бройля: \[v_\phi=\dfrac{w}{k}=\dfrac{\varepsilon}{p}=\dfrac{mc^2}{mv}=\dfrac{c^2}{v}\]

Фазовая скорость частицы сама по себе не несёт конкретной информации о скорости частицы (ведь получается, что фазовая скорость больше скорости света). Информацию о скорости частицы несёт волновой пакет.

Групповая скорость волн де Бройля: \[v_\phi=\dfrac{dw}{dk}=\dfrac{d\varepsilon}{dp}=[d\varepsilon=vdp]=v\]

Групповая скорость волн де Бройля равна скорости частицы.

Связь между импульсом частицы и энергией в релятивистском случае:

\[\left(\dfrac{\varepsilon}{c}\right)^2 - p^2 = m_0^2c^2\]

Подставим в это выражение:

\[\varepsilon=\hbar w \text{ и } p = \dfrac{hv}{c}=\dfrac{\hbar w}{c} = \dfrac{2\pi \hbar}{\lambda}=\hbar k\]

Тогда получаем:

$$
	\left(\dfrac{\hbar w}{c}\right)^2 - (\hbar k)^2  =  m_0^2c^2 
$$
$$
	\left(\dfrac{1}{c}\right)^2 - \left(\dfrac{k}{\omega}\right)^2  =  \dfrac{m_0^2c^2}{\hbar^2} \cdot \dfrac{1}{\omega^2}
$$

Фазовая скорость волн де Бройля $v_{\text{ф}} = \dfrac{w}{k}$ зависит от частоты w --- в этом и состоит \textbf{дисперсия}.

Частица, если бы она представляла волновое образование, была бы неустойчива и быстро распалась бы, что не соответствует действительности. Таким образом, частица не может быть волновым пакетом, образованным из волн де Бройля. Т.е. волны, которые образуют волновой пакет, каждая из них движется со своей фазовой скоростью, которая зависит от частоты (или, что тоже самое, длины волны). В итоге, эти волны, составляющие волновой пакет с течением времени перемешиваются и волновой пакет расплывается. Посчитали его время жизни, оказалось, что это $10^{-3}$ секунды. Т.е. получается, что время жизни электрона очень моленькое, однако другие опыты показывают, что время жизни жэлектрона может быть достаточно большим. Т.е. пока мы так и не понимаем, что такое волна де Бройля.

Рассмотрим следующий мысленный эксперимент. Направим на преграду с двумя узкими щелями параллельный пучок электронов, обладающих одинаковой кинетической энергией. За преградой поставим фотопластинку. Будем по очереди закрывать то одну, то другую щель и смотреть распределение максимумов и минимумов дифракции. Так вот волна де Бройля проходит через две щели, в то время как \textbf{про сами электроны} в этой волне де Бройля мы не можем сказать, через какую щель (1 или 2) прошёл каждый конкретный электрони в этой волне де Бройля. Этот опыт показывает, что понятие траектории у электрона отсутствует. Так как есть дифракция, мы просто можем сказать, в какой место в среднем пришло больше или меньше электронов. Волны де Бройля фактически имеют статистический смысл. \textbf{Вероятность попадания электронов на фотопластинку в тех местах пространства больше, где интенсивность волн де Бройля больше.}

\mysubsection{Соотношение неопределённостей Гейзенберга}

Волновой пакет: множество синусоид одинаковой амплитуды, волновые числа которых последовательно возрастают от синусоиды  к синусоиде на одну и ту же величину.

В точке $x$ (это в начале волнового пакета, конец волнового пакета будет в точке $x+\Delta x$) фазы волн меняются от $kx$ до $(k+\Delta k)x$, т.е. на $x\Delta k$. Каждую отдельную волну в волновом пакете мы рассчитываем по формуле:

\[\xi=A\cos(\omega t - kx)\]

где у каждой отдельно взятой волны k меняется на \textbf{одинаковую} величину $\Delta k'$, а разница между самым большим и самым маленьким $k$ равняется $\Delta k$. Ширина волнового пакета $\Delta x$. Если $x\Delta k =2\pi$, то в этой точке все синусоиды взаимно гасят друг друга. Но мы продолжаем рассматривать только волны с одинаковой амплитудой. Точка $x+\Delta x$ --- ближайшая точка, где происходит следующее гашение. В точке $x + \Delta x$ фазы волн меняются от $k(x +\Delta x)$ до $(k + \Delta k)(x + \Delta x)$. Записываем разность фаз в точке $x + \Delta x$:

\[(k + \Delta k)(x + \Delta x) - k(x+\Delta x) = x\Delta k + \Delta x \Delta k =2\pi + \Delta x\Delta k\]

Получается, что следующее ближайшее гашение будет, когда $\Delta x\Delta k = 2\pi$. Если же амплитуды не одинаковые, то $\Delta x\Delta k \geqslant 2\pi$. Если перейти к волновому пакету, составленному из волн де Бройля, то нужно вместо $k$ написать импульс $p=\hbar k$. Тогда импульс частиц, составленных из волн де Бройля будет изменяться от $p=\hbar k$ до $p + \Delta p=\hbar(k+\Delta k)$. Тогда аналогично получаем условие, про котором происходит гашение волн волнового пакета справа и слева (мы берём проекцию по оси $x$):

\[\Delta x\Delta p_x \geqslant 2\pi \hbar\] или

\[ \Delta x \Delta p_x \geqslant h\]

Любое из двух последних выражений (так как это одно и тоже) и называется соотношением Гейзенберга. Запишем его для всех координат:

\[ \Delta x \Delta p_x \geqslant h\]
\[ \Delta y \Delta p_y \geqslant h\]
\[ \Delta z \Delta p_z \geqslant h\]

О чём говорит это соотношение?

Рассмотрим два предельных случая. Если неопределённость импульса равна 0, это означает, что импульс принимает определённое значение, а это одначает, что частица движется с одной строго определённой скорость. Но тогда из соотношения неопределённости следует, что неопределённостькоординаты равна бесконечности. А это означает, что о величине координаты, т.е. о том, где находится частица, ничего сказать нельзя. Т.е. частица движется с постоянной скоростью, но где она находится, мы сказать не можем.

Второе. Пусть неопределённость координаты равна 0. Это означает, что известно, где нахолдится частица, т.е. она имеет определённую координату. А это означает, что неопределённость импульса равна бесконечности, т.е. если мы знаем, где находится частица, мы ничего не можем сказать о том, какую скорость она имеет (т.е. скорость у неё может быть любая).

*Если неопределённость скорости больше или равна заданной скорости, то к такому объекту можно применять только законы квантовой механике (аналогично для импульса и координаты).

Также существует соотношение неопределённостей для энергии и времени:

\[\Delta t\Delta \omega \geqslant 2\pi\]

Т.е. ограниченный по времени процесс не может быть монохроматическим (т.е. не может происходить с одинаковой частотой).
Далее, выразим частоту через энергию с помощью такого соотношения:

\[E=\hbar \omega\]

Тогда аналогично получаем:

\[\Delta E\Delta t \geqslant h\]

Получается, что чем короче время существования какого-то состояния, тем с меньшей определённостью можно говорить об энергии этого состояния. Чем с большей точностью определено время, тем больший разброс по энергиям. Поэтому, кстати, тут не имеет смысла говорить о законе сохранения энергии, т.е. например, те частицы, которые имеют мальнеое время жизни и в конечном итоге распадаются, об их энергии (и тем более о сохранении энергии) ничего сказать нельзя.

Рассмотрим энергетические уровни атома. Пусть $h\nu = E_2 - E_1$. Тогда исходя из соотношения $\Delta E\Delta t \geqslant h$ и того, что время жизни электрона на орбите достаточно большое, получаем, что $\Delta E \rightarrow 0$. А это значит, что энергенически уровни в этоме действительно представлены тонкими <<полосочками>>.

А теперь рассмотрим возбуждённое состояние. Пусть: $\Delta t =\tau$. Тогда из соотношения неопределённостей получаем, что $\Delta E \neq 0$. Получается, что в возбуждённом состоянии происходит \textit{естественное уширение} линии. Т.е. переход совершается не точно на эту линюю, а на интервал, поэтому излучаются немного разные частоты и, как следствие, происходит размывание спектральной линии. 

\mysection{Волновые функции и операторы}

\mysubsection{Волновая функция. Вероятность нахождения микрочастицы.}

Из-за того, что частицы обладают одновременно и корпускулярными и волновыми свойствами, возникают определённые статистические закономерности. Математический аппарат квантовой механики отличается от классической механики.

Основу математичесмкого аппарата квантовой механики составляет утверждение, что \textbf{описание состояния системы очсуществляется заданием определённой (вообще говоря, комплексной) функцией $\Psi(r,t)$}, где $r$ --- это радиус вектор (выражает зависимость (пси)-функции от координаты), а $t$ --- это время.

$\Psi$ ---волновая функция системы, или просто $\Psi$ (пси)-функция.

Электрон, вообще говоря, тоже система. $\Psi(r,t)$ --- принципиально комплексная (т.е. это означает, что любые физические свойства выражаются через всю (пси)-функцию, а не только через её действительную или мнимую часть). Т.к. (пси)-фунция --- величина комплексная, то она не измерима, так как измеримы только вещественные величины. Эта функция не имеет физического смысла, она описывает только состояние ссистемы.

Согласно статистической интерпритации волн де Бройля, частицы попадают в те места пространства, где интенсивность волн де Бройля максимальна. Следовательно, интенсивность можно было бы заменить волнами вероятности. Но тогда как и любая волна, эта вероятность в какой-то момент времени должна быть отрицательной, чего быть не может. Тогда Борн предположил, что по волновому закону изменяяется не сама вероятность, а амплитуда вероятности. Из классической физики мы знаем, что интенсивность пропорциональна квадрату амплитуды (например для света).

В квантовой ж механие физический смысл имеет квадрат модуля (пси)-функции:

\[|\Psi |^2 = \overline \Psi \cdot \Psi\]

Это вероятность обнаружить частицу в каком-то месте пространства. Рассмотрим несколько примеров:

\begin{itemize}
	\item \textbf{Пример 1}. Де Бройль сопоставил частице, движущейся со скоростью $v$, плоскую монохроматическую волну: $\Psi = \Psi_0 e^{i(kr-\omega t)}$ (обратим внимание на то, что в отличии от того, как мы записывали волны в классической механике, тут мы \textbf{не из} $wt$ что-то вычитаем, а вычитаем само $wt$, хотя это не особо важно, просто такое обозначение букв). Получаем:
	
	\[|\Psi|^2=\overline\Psi_0 \Psi_0 = const\]
	
	т.е. равновероятностно обнаружить частицу в любом месте пространства.
	
	\item \textbf{Пример 2} Посмотрим, будет ли наблюдаться интерференция волн, если каждая из них существенно комплексна. Пусть две волны де Бройля представляются выражениями:
	
	\[\Psi_1=\Psi_0 e^{-i(kr - \omega t)}\]
	\[\Psi_2=\Psi_0 e^{-i(kr - \omega t + \delta(r))}\]
	
	где $\delta(r)$ - разность фаз между двумя волнами.
	При наложении волн получаем: $\Psi = \Psi_1 + \Psi_2$.
	Поэтому вероятность обнаружить частицу в каком-то месте пространства считается так:
	
	\begin{gather*}
		\overline\Psi_1 = \overline\Psi_0 e^{i(kr-\omega t)} \\
		\overline\Psi_2=\overline\Psi_0 e^{i(kr - \omega t + \delta(r))} \\
		\Psi = \Psi_1 + \Psi_2 \\
		\overline\Psi = \overline\Psi_1 + \overline\Psi_2
	\end{gather*}
	Далее:
	\[|\Psi|^2=\overline\Psi \Psi=(\overline\Psi_1 + \overline\Psi_2)(\Psi_1 + \Psi_2)\]
	Раскрываем скобки:
	\begin{gather*}
	|\Psi|^2=\overline\Psi_1 \Psi_1 + \overline\Psi_1 \Psi_2 + \overline\Psi_2 \Psi_1 + \overline\Psi_2 \Psi_2 = \overline\Psi_0 \Psi_0 (e^0 + e^{-i\delta} + e^{i\delta} + 1) =
	\end{gather*}
	Раскрываем экспоненты по формуле Эйлера:
	\begin{gather*}
		=\overline\Psi_0 \Psi_0(2 + \cos\delta -i\sin\delta + \cos\delta + i\sin\delta) = 2\overline\Psi_0 \Psi_0(1+cos\delta) 
	\end{gather*}
	Т.е. мы получаем, что вероятность $(|\Psi|^2)$ зависит от разности фаз $(\delta)$, а это и есть интерференционный член (интерференционное слагаемое).
\end{itemize}

$|\Psi|^2$ определяет вероятность нахождения частицы в том или ином месте пространства. Чтобы немного сузить это понятие, введём новое $dP=|\Psi|^2dV$ --- вероятность обнаружения частицы в элемпенте объёма пространства dV.

Для того, чтобы волновая функция была объективной характеристикой состояния частицы, на неё должны накладываться некоторые ограничения:

\begin{enumerate}
	\item Конечность, т.к. вероятность не может быть больше 1.
	\item Однозначность, т.к. вероятность не может быть не однозначной.
	\item Непрерывность, т.к. вероятность не может изменяться скачком.
	\item Должно выполняться условие нормировки: $\int|\Psi|^2dV=1$ --- т.е. во всём пространстве (объёме) частица будет обнаружена с достоверностью. Соответственно, чтобы это выполнялось, интеграл должен сходиться. 
\end{enumerate}

\mysubsection{Собственные значение данной физической величины}

Пусть $f$ --- некоторая физическая величина. \textit{Собственными значениями данной физической величины} называются значения, которые может принимать данная физическая величина. Т.е. мы измеряем физическую величину и при этом с ненулевой вероятностью получаем то или иное значение. Собственные значение могут быть как непрерывными (например, координата), так и дискретными (например, энергия электрона в теории Бора).

Пусть $f$ принимает дискретные значения. Собственные значения физической величины обозначим как $f_n$, $n=1,2\dots$. При измерении величины $f$ мы с ненулевой вероятностью получаем одно из собственных значений.

\mysubsection{Собственная функция данной физической величины}

Как мы знаем, \textit{волновая функция} описывает состояние системы. \textit{Собственная} же функция $\Psi_n$ данной физической величины описывает состояние системы, находясь в котором физическая величина $f$ принимает одно, строго определённое значение $f_n$. Т.е. если система находится в состоянии, которое описывается собственной функцией $\Psi_1$, то при измерении физической величины мы \textbf{всегда} получаем одно единственное значение $f_1$.

\mysubsection{Принцип суперпозиции волновых функций}

Это важнейший принцип в квантовой механике. В классической механие тоже есть принцип суперпозиции (например, электрических полей). Причём по принципу суперпозиции электрических полей эти поля складваются. В квантовой механике точно также складываются волновые функции, которые описывают состояние системы (но физического смысла эти функции не несут). В классической механике тоже интенсивность пропорциональна квадрату амплитуды, но складываются не интенсивности. Даже когда мы рассмативали сложение колебаний одного направления с разными частотами, мы использовали метод вращающегося вектора и интенсивность колебания --- это как раз квадрат амплитуды, а квадрат амплитуды мы получали как раз в результате сложения двух напряжённостей (в результате сложения двух полей). Здесь всё аналогичным образом, только напряжённость в механике имеет фихзический смысл, а тут (пси)-функция физического смысмла не несёт, но описывает состояние системы.

Поэтому, если какая-то физическая величина имеет собственные функции $\Psi_1(r,t), \Psi_2(r,t), \dots \Psi_n(r,t)$, то всякая функция, представленная в виде суперпозиции этих функций с некоторыми коэффициентами ($C_1, C_2, \dots C_n$, которые постоянные, но, вообще говоря, комплексные):

\[\Psi=\sum\limits_n C_n \Psi_n\]

то эта (пси)-функция тоже описывает некоторое состояние системы, но находясь в этом состоянии, физическая величина $f$ \textbf{уже не имеет определённого значения}. Т.е. при измерении величины $f$ будет получатся либо значение $f_1$, либо $f_2$ и т.д. Причём вероятности появления этих значений равны квадратам модулей коэффициентов $|C_1|^2, |C_2|^2 \dots |C_n|^2$. Ну а поскольку это вероятности, то сумма этих вероятностей должна быть равна 1, поэтому (всегда выполняется для нормированных $\Psi_n$):

\[\sum\limits_n |C_n|^2 = \sum\limits_n \overline C_n \cdot C_n = 1\]

Поскольку в состоянии, описываемом $\Psi$ -функцией, физическая величина $f$ уже не принимает определённого значения, то среднее значение физической величины $<f>$ в состоянии, описываемом функцией $\Psi = \sum\limits_n C_n \Psi_n$ равно:

\[<f> = \sum\limits_n |C_n|^2 \cdot f_n = \sum\limits_n \overline C_n \cdot C_n \cdot  f_n\] 

Аналогия: эта формула рассчитывается также, как рассчитывается накопленная оценка по какому-либо предмету.

\begin{center}
	\textbf{Пример}
\end{center}

Пусть физическая величина $f$ может принимать всего два значения: $f_1$ и $f_2$, где 
$f_1$ и $f_2$ являются собственными значениями данной физической величины $f$. Тогда $\Psi_1(r,t)$ и $\Psi_2(r,t)$ - собственные функции данной физической величины $f$. Получаем:

\begin{center}
\begin{tabular}{  c | c | c  }
	Состояние системы & $\Psi_1(r,t)$ & $\Psi_1(r,t)$ \\
	 & $\updownarrow$ & $\updownarrow$ \\
	 Физическая величина & $f_1$ & $f_2$
\end{tabular}
\end{center}

Ещё раз скажем, что показано в этой таблице. Если мы находимся в состоянии, описываемом функцией $\Psi_1(r,t)$, то при измерении величины $f$ мы будем \textbf{всегда} получать $f_1$. Аналогично для третьего столбца.

Далее, принцип суперпозиции гласит, что находясь в состоянии $\Psi$, где

\[\Psi = C_1\Psi_1 + C_2\Psi_2\]

мы будем получать значение физической величины $f$ равные либо $f_1$, либо $f_2$. Здесь существенное отличие от классической физике: если мы в классической физике складываем две волны, то мы получаем третью волну, а здесь новая физическая величина $\Psi$ имеет после сложения $C_1\Psi_1 + C_2\Psi_2$ всего два значения --- $f_1$ и $f_2$. Поэтому при измерении $f$, с вероятностью $|C_1|^2$ мы получаем $f_1$. Аналогично для $f_2$. К тому же:

\begin{gather*}
	|C_1|^2 + |C_2|^2 = 1 \\
	<f> = |C_1|^2 \cdot f_1 + |C_2|^2 \cdot f_2
\end{gather*}

\mysubsection{Свойства $\Psi$-функции}

Состояние системы полностью описывается $\Psi$-функцией. Напомним, что $\Psi$-функция, являясь комплексной величиной, не имеет физического смысла. Физический смысл имеет квадрат модуля $\Psi$-функции ($|\Psi|^2 = \overline \Psi \Psi$), опрнеделяющий вероятность нахождения частицы в той или иной облати пространства, и являющийся измеримой величиной. Повторим сами свойства, которые уже были перечислены ранее, и добави одно новое:

\begin{enumerate}
	\item Конечность
	\item Однозначность
	\item Непрерывность
	\item Условие нормировки
	\item (новое) Собственные волновые функции должны быть ортогональны.
\end{enumerate}

\mysubsection{Ортогональность собственных функций}

Докажем это. Пусть $\Psi(r,t), \Psi(r,t) \dots$ --- собственный функции некоторой физической величины $f$. Тогда согласно принципу суперпозиции $\Psi = \sum\limits_n C_n \Psi_n$, где $|C_n|^2$ --- вероятность соответствующего значения $f_n$ величины $f$ в состоянии, описываемом волновой функцией $\Psi$.

Волновая функция является нормированной, т.е.:

\[\int \overline\Psi \Psi dV = 1\]

Это с одной стороны. С другой, сумма вероятностей всех возможных значений $f_n$ должна быть равна единице:

\[\sum\limits_n \overline C_n C_n = 1\]

Так как правые части двух последих выражений одинаковы и равны единице, следовательно:

\[\sum\limits_n \overline C_n C_n = \int \overline \Psi \Psi dV \eqno(1)\]

Но $\Psi$-функцию можно разложить по собственным функциям:

\begin{gather*}
	\Psi = \sum\limits_n C_n \Psi_n \\
	\overline \Psi = \sum\limits_n \overline C_n \overline \Psi_n
\end{gather*}

Подставим $\overline \Psi$ в интеграл в формуле (1):

\begin{gather*}
\int \overline \Psi \Psi dV = \int \left(\sum\limits_n \overline C_n \overline \Psi_n\right) \cdot \Psi dV = \int \left(\sum\limits_n \overline C_n \overline \Psi_n \Psi\right)dV = \\
= \sum\limits_n \left(\int \overline C_n \overline \Psi_n \Psi dV\right) = \sum\limits_n \overline C_n \left(\int\overline\Psi_n \Psi dV\right) 
\end{gather*}

Все эти преобразования можно делать, т.к. $\Psi$ не зависит от $n$, а коэффициенты $C_n$ не зависят от координат. Теперь сравним левую часть из формулы (1) и то, что мы получили выше:

\[\sum\limits_n \overline C_n C_n = \sum\limits_n \overline C_n \left(\int\overline\Psi_n \Psi dV\right)\]

Очевидно получаем, что:

\[C_n = \int\overline\Psi_n \Psi dV\]

Подставим в это выражение $\Psi = \sum\limits_m C_m \Psi_m$:

\begin{gather*}
	C_n = \int \overline\Psi_n \Psi dV = \int \left(\sum\limits_m C_m \Psi_m\right) \cdot \overline\Psi_n dV = \int \left(\sum\limits_m C_m \Psi_m \overline\Psi_n\right)dV = \\
	= \sum\limits_m C_m \left(\int \overline\Psi_n \Psi_m dV\right)
\end{gather*}

Таким образом, получили:

\[C_n = \sum\limits_m C_m \int \overline\Psi_n \Psi_m dV\]

Отсюда следует, что собственные функции должны удовлетворять условиям (тут не совсем понятный переход, нужно пояснить это из интернета):

\[\int \overline\Psi_n \Psi_m dV = \delta_{nm}\]

\begin{equation*}
\text{где } \delta_{nm} = 
\begin{cases}
1, \text{ при } n=m \\
0, \text{ при } n \ne m
\end{cases}
\end{equation*}

Согласно принципу суперпозиции, любую волновую функцию какого-то состояния частицы можно разложить по собственным функциям данной физической величины:

\[\Psi = \sum\limits_n C_n \Psi_n\]

$|C_n^2|$ определяет вероятность соответствующего значения $f_n$ величины $f$ в состоянии, описываемом волновой функцией $\Psi$.

По определению, среднее значение физической величины $<f>$ в данном состоянии может быть представлено в виде:

\[<f>=\sum\limits_n f_n |C_n|^2\]

В квантовой механике утверждается, что \textbf{каждой физической величине $f$ ставится в соответствие оператор $\hat f$}. Оператор --- это как бы некоторая математическая операция, которая переводит функцию одних переменных в другую функцию этих же переменных. Определим оператор $\hat f$ следующим образом:

\[<f> = \int \overline \Psi (\hat f \Psi) dV\]

\mysubsection{Свойства операторов, соответствующих физическим величинам}

\begin{enumerate}
	\item Покажем, что каждой физической величине в квантовой механике ставятся в соответствие \textbf{линейный оператор}. По определению, линейный оператор --- это: $\hat f(a\varphi + b\psi) = a\hat f \varphi + b\hat f \psi$.
	
	Среднее значение физической величины $<f>$ в состоянии, которое описывается $\Psi$-функцией:
	
	\[\Psi = \sum\limits_n C_n \Psi_n\]
	
	определяется выражением (это с одной стороны):
	
	\[<f> = \sum\limits_n f_n |C_n|^2\]
	
	С другой стороны, среднее значение физической величины $<~f>$ можно представить в виде:
	
	\[<f> = \int \overline \Psi (\hat f \Psi) dV \eqno(2)\]
	
	Вспомним, что волновая функция является нормированной, т.е.:
	
	\[\int \overline\Psi \Psi dV = 1\]
	
	Вспомним, что сумма вероятностей всех возможных значений $f_n$ должна быть равна единице, т.е.:
	
	\[\sum\limits_n \overline C_n C_n = 1\]
	
	Тогда получаем, что:
	
	\[\sum\limits_n \overline C_n C_n = \int \overline \Psi \Psi dV \]
	
	Теперь подставим разложение $\Psi = \sum\limits_n C_n \Psi_n$ в интеграл, строчкой выше:
	
	\begin{gather*}
		\int \overline\Psi \Psi dV = \int \overline\Psi \left(\sum\limits_n C_n \Psi_n\right) \cdot dV = \int \left(\sum\limits_n \overline\Psi C_n \Psi_n\right) dV = \\
		= \sum\limits_n \left(\int \overline \Psi C_n \Psi_n dV\right) = \sum\limits_n C_n \left(\int \overline \Psi \Psi_n dV\right)
	\end{gather*}
	
	Получили: $\int \overline \Psi \Psi dV = \sum\limits_n C_n \left(\int \overline \Psi \Psi_n dV\right)$
	
	Подставим полученное значение интеграла в $\sum\limits_n \overline C_n C_n = \int \overline \Psi \Psi dV$, то есть:
	
	\[\sum\limits_n \overline C_n C_n = \sum\limits_n C_n \int \overline \Psi \Psi_n dV\]
	
	В итоге:
	
	\[\overline C_n = \int \overline \Psi \Psi_n dV\]
	
	Теперь то, что мы получили выше, подставляем в $<f> = \sum\limits_n f_n|C_n|^2$, получим (тут также пользуемся свойством, что сумма интегралов равна интегралу суммы):
	
\begin{gather*}
<f> = \sum\limits_n |C_n|^2 \cdot f_n = \sum\limits_n \overline C_n \cdot C_n \cdot f_n = \sum\limits_n C_n \cdot f_n \cdot \left(\int \overline \Psi \Psi_n dV\right) = \\
= \sum\limits_n \left(\int C_n f_n \overline \Psi \Psi_n dV\right) = \int \left(\sum\limits_n C_n f_n \overline \Psi \Psi_n\right) dV = \int \overline \Psi \left(\sum\limits_n C_n f_n \Psi_n\right) dV
\end{gather*}

В итоге получаем:

\[<f> = \int \overline \Psi \left(\sum\limits_n C_n f_n \Psi_n\right) dV\]

Сравним получившееся выражение с выражением (2). Оно выглядело так:	

\[<f> = \int \overline \Psi (\hat f \Psi) dV \]

Получим:

\[\hat f \Psi = \sum\limits_n C_n f_n \Psi_n \eqno(3)\]

Если $\Psi = \Psi_n$, то в выражении выше все $C_n$, кроме одного, равны нулю, а само $C_n$, получается, равно единице и тогда:

\[\hat f \Psi = f_n \Psi_n \eqno(4)\]

Подставляя (4) в (3), получим:

\[\hat f \Psi = \sum\limits_n C_n f_n \Psi_n = \sum\limits_n C_n \hat f \Psi_n\]

С другой стороны, согласно принципу суперпозиции $\Psi = \sum\limits_n C_n \Psi_n$, поэтому получаем:

\[\hat f \Psi = \hat f \sum\limits_n C_n \Psi_n\]

Сопоставляя последние два выражения, получаем:

\[\hat f \sum\limits_n C_n \Psi_n = \sum\limits_n C_n \hat f \Psi_n \Rightarrow \text{ оператор $\hat f$ --- линейный}\]
	
	\item \textbf{Уравнение на собственные функции и собственные значения физической величины}.
	
	Из уравнения $\hat f \Psi = \sum\limits_n C_n f_n \Psi_n$ видно, что если функцией $\Psi$ является одна из собственны функций $\Psi_n$ (так, что все $C_n$, кроме одного, равны нулю), то в результате воздействия на неё оператора $\hat f$ эта функция просто умножается на соответствующее собственной значение $\hat f_n$:
	
	\[\hat f \Psi_n = f_n \Psi_n\]
	
	Таким образом, собственные функции $\Psi_n$ Данной физической величины $f$ являются решениями уравнения:
	
	\[\hat f \Psi = f \Psi \eqno(5)\]
	
	Функции, удовлетворяющие уравнению $\hat f \Psi = f \Psi$ и являющиеся рещением этого уравнения, называются \textit{собственными функциями оператора} $\hat f$. Числа $f$, при которых уравнение (5) имеет решения, называются \textit{собственными значениями оператора} $\hat f$.
	
	В квантовой механике принимаетсмя, что при измерении физической величины $f$ могут получаться только собственные значения $f_n$ соответствующего ей оператора $\hat f$. Таким образом, уравнение (5) является \textbf{уравнением на собственные функции и собственные значения данной физической величины}. В общем случае при имерении $f$ мы можем получить любое из собственных значений с той или иной вероятностью.
	
	\item \textbf{Эрмитовы операторы}
	
	Собственные значения вещественной физической величины и её среднее значение в любом состоянии являются вещественными, то есть:
	
	\[f=\overline f\]
	
	Среднее значение физической величины f равно:
	
	\[<f> = \int \overline \Psi (\hat f \Psi) dV\]
	
	Поскольку среднее значение вещественно, то оно равно комплексно-сопряжённому среднего:
	
	\[<f> = \overline{<f>} \eqno(6)\]
	
	В это выражение мы подставляем среднее значение физической величины по определению:
	
	\[\int \overline\Psi (\hat f \Psi) dV = \overline{\left(\int \overline \Psi (\hat f \Psi) dV\right)}\]
	
	Или (что тоже самое, т.е. просто всё заменяем на комплексно сопряжённые числа):
	
	\[\int \overline \Psi (\hat f \Psi) dV = \int \Psi (\overline{\hat f} \overline \Psi) dV \eqno(7)\]
	
	Здесть $\overline{\hat f}$ --- оператор, комплексно сопряжённый с $\hat f$.
	
	Соотношение (7) записано \textbf{только для операторов, соответствующих физическим величинам (определили в (6))}, которые всегда вещественны.
	
	Однако, для произвольного оператора $\hat f$ можно указать транспонированный с ним оператор $\widetilde{\hat f}$, определяемый как:
	
	\[\int \Phi (\hat f \Psi) dV = \int \Psi(\widetilde{\hat f} \Phi) dV\]
	
	где $\Phi, \Psi$ --- две различный функции. Такое соотношение верно не только для физики, но и для любого оператора в математике.
	
	Если в качестве функции $\Phi$ выбрать функцию $\overline \Psi$, то:
	
	\[\int \overline \Psi (\hat f \Psi) dV = \int \Psi (\widetilde{\hat f} \overline \Psi) dV \eqno(8)\]
	
	Сопоставляя выражения (7) и (8) получаем (т.к. левые части у них равны), что:
	
	\[\widetilde{\hat f} = \overline{\hat f}\]
	
	Операторы, удовлетворяющие условию выше, называются эрмитовыми.
	
	Таким образом, \textbf{операторы, соответствующие физическим величинам, должны быть эрмитовыми}.
	
	Покажем, что собственные значения эрмитова оператора действиетльны.
	
	Пусть $\Psi_1, \Psi_2,  \dots , \Psi_n$ --- собственные функции физической величины $f$, и оператор, соответствующий данной физической величине $\hat f$ --- эрмитов.
	
	Тогда, если системы находятся в состоянии, описываемом $\Psi_n$, то уравнение на собственные функции и собственные значения:
	
	\[\hat f \Psi_n = f_n \Psi_n \eqno(9)\]
	
	Возьмём комплексно сопряжённое от этого уравнения:
	
	\[\overline{\hat f} \overline \Psi_n = \overline f_n \overline \Psi_n \eqno(10)\]
	
	Домножаем (9) на $\overline \Psi_n$ слева и потом всё интегрируем по объёму:
	
	\[\int \overline \Psi_n \hat f \Psi_n dV = \int \overline \Psi_n f_n \Psi_n dV\]
	
	\[\int \overline \Psi_n \hat f \Psi_n dV = f_n \int \overline \Psi_n \Psi_n dV\]
	
	Но $\int \overline \Psi_n \Psi_n dV = 1$ по условию ортогональности, которое доказали раньше. Тогда получаем, что:
	
	\[\int \overline \Psi_n \hat f \Psi_n dV = f_n\]
	
	Аналогично умнеожем (9), но только на $\Psi_n$ и тоже интегрируем по объёму:
	
	\[\int \Psi_n \overline{\hat f} \overline \Psi_n dV = \int \Psi_n \overline f_n \overline \Psi_n dV\]
	
	\[\int \Psi_n \overline{\hat f} \overline \Psi_n dV = \overline f_n \int \Psi_n \overline \Psi_n dV\]
	
	Но $\int \overline \Psi_n \Psi_n dV = 1$ по условию ортогональности, которое доказали раньше. Тогда получаем, что:
	
	\[\int \Psi_n \overline{\hat f} \overline \Psi_n dV = \overline f_n\]
	
	Но поскольку $\hat f$ --- эрмитов оператор, т.е. $(\overline{\hat f} = \widetilde{\hat f})$, то используя соотношение выше получаем:
	
	\[\overline f_n = \int \Psi_n \overline{\hat f} \overline \Psi_n dV = \int \Psi_n \widetilde{\hat f} \overline \Psi_n dV = \int \overline \Psi_n \hat f_n dV = f_n\]
	
	Предпоследнее равенство получаем, т.к. транспонированный оператор меняет местами функции (см. формулу (8)). Получаем:
	
	\[\overline f_n = f_n\]
	
	Т.е. собственные значения эрмитового оператора действительны.
	
	В итоге:
	
	\begin{itemize}
		\item Каждой физической величине в квантовой механике можно поставить в соответствие линейный эрмитов оператор.
		\item Решая уравнение на собственные функции и собственные значения данной физической величины
		
		\[\hat f \Psi = f \Psi\]
		
		можно найти собственные функции и собственные значения данной физической величины.
	\end{itemize}
	
\end{enumerate}

\mysubsection{Условия возможности одновременного измерения различных физических величин}

Пусть $\hat f$ и $\hat g$ два оператора, каждому из которых соответствует свой спектр собственных значений. Что значит, что физические величины одновременно измеримы? Это означает, что $\Psi_n$ является собственной функцией как оператора $\hat f$, так и $\hat g$. Тогда уравнение на собственные функции и собственные значения для собственной функции $\Psi_n$:

\[\hat f \Psi_n = f_n \Psi_n\]

\[\hat g \Psi_n = g_n \Psi_n\]

Здесь $f_n$ и $g_n$ --- собственные значения операторов $\hat f$ и $\hat g$ в одном и том же состоянии $\Psi_n$.

Умножим первое равенство слева на оператор $\hat g$:

\[\hat g \hat f \Psi_n = \hat g (\hat f \Psi_n) = \hat g (f_n \Psi_n) = f_n (\hat g \Psi_n) = f_n g_n \Psi_n\]

Предпоследнее равенство получаем исходя из того, что оператор $\hat g$ линейный, а $f_n$ --- постоянный коэффициент.

Аналогично

\[\hat f \hat g \Psi_n = g_n f_n \Psi_n\]

Отсюда

\[(\hat f \hat g - \hat g \hat f) \Psi_n = 0\]

Но! Соотношение $(\hat f \hat g - \hat g \hat f) = 0$ может выполняться не всегда, т.к. $\Psi_n$ --- не произвольная функция, а льшь одна из собственных функций операторов $\hat f$ и $\hat g$. Т.е. нужно, чтобы каждая собственная функция оператора $\hat f$ являлась так же собственной функцией оператора $\hat g$ и наоборот. Тогда, т.к.

\[\Psi = \sum\limits_n C_n \Psi_n\]

то

\[(\hat f \hat g - \hat g \hat f) = 0\]

и тогда операторы $\hat f$ и $\hat g$ коммутируют.

$(\hat f \hat g - \hat g \hat f)$ называют коммутатором и обозначают $\{\hat f, \hat g\}$. И если две физические величины одновременно измеримы, то коммутатор равен 0. Можно доказать и обратное. Если комутатор отличен от 0, то две физхические величины невозможно одновременно измерить.

\mysection{Основные операторы квантовой механики}

\mysubsection{Оператор координат}

Оператор координат --- сама координата $(\hat x = x)$.

Уравнение на собственные функции и собственные значения выглядит так:

\[\hat x \Psi = x\Psi\]

Подставляя в это уравнение $(\hat x = x)$, получаем:

\[x \Psi = x\Psi\]

Понятно, что этому уравнению удовлетворяет любая $\Psi$ - функция и любая координата, т.е. спектр собственных значений (т.е. значений, которые может принимать координата) непрерывный.

\mysubsection{Оператор проекции импульса}

Оператор проекции импульса на ось $O_x$ определяется выражением:

\[\hat p_x = -i \hbar \dfrac{\partial}{\partial x} = \dfrac{\hbar \partial}{i\partial x}\]

Аналогично для остальных осей:

\[\hat p_y = -i \hbar \dfrac{\partial}{\partial y} = \dfrac{\hbar \partial}{i\partial y}\]

\[\hat p_z = -i \hbar \dfrac{\partial}{\partial z} = \dfrac{\hbar \partial}{i\partial z}\]

Найдём собственные функции оператора проекции импульса. Уравнения на собственные функции и собственные значения оператора проекции импульса на ось $O_x$ имеет вид:

\[\hat p_x \Psi(x) = p_x \Psi(x)\]

\[-i\hbar \dfrac{\partial \Psi}{\partial x} = p_x \Psi\]

Решаем это уравнение методом разделения переменных:

\begin{gather*}
	\dfrac{d\Psi}{\Psi} = -\dfrac{1}{i\hbar} p_x dx \\
	\dfrac{d\Psi}{\Psi} = \dfrac{i}{\hbar} p_x dx \\
	\int\limits_{\Psi_0}^{\Psi} \dfrac{d\Psi}{\Psi} = \dfrac{i}{\hbar} p_x \int\limits_0^x dx \\
	ln \left(\dfrac{\Psi}{\Psi_0}\right) = \dfrac{i}{\hbar} p_x x \\
	\Psi = \Psi_0 e^{\left(\dfrac{i}{\hbar}p_x x\right)} = \Psi_0 e^{ik_x x}
\end{gather*}

Получаем, что $e^{ik_x x}$ --- собственная функция оперпатора $\hat p_x$. Она описывает состояние частицы, движущейся без внешних воздействий вдоль оси $O_x$ с постоянной скоростью (импульсом).

\mysubsection{Оператор вектора импульса}

Импульс частицы в классической механике равен:

\[\vec p = p_x \vec i + p_y \vec j + p_z \vec k\]

В квантовой механике физической величине сопоставляем оператор:

\[\hat{\vec p} = \hat p_x \vec i + \hat p_y \vec j + \hat p_z \vec k\]

\[\hat{\vec p} = -i\hbar \left(\dfrac{\partial}{\partial x} \vec i + \dfrac{\partial}{\partial y} \vec j + \dfrac{\partial}{\partial z} \vec k\right) = -i\hbar \vec \nabla\]

\[\hat \vec p = -i\hbar \vec \nabla = \dfrac{i}{\hbar} \vec \nabla\]

Покажем, что координата $x$ и проекция импульса $p_x$ не могут быть одновременно измерены. Для этого вычислим комутатор $\{\hat x, \hat p_x\} = \hat x \hat p_x - \hat p_x \hat x$

\begin{gather*}
	\hat x \hat p_x \Psi = \hat x (\hat p_x \Psi) = \hat x \cdot \left(- i\hbar \dfrac{\partial \Psi}{\partial x}\right) = -i\hbar x \dfrac{\partial \Psi}{\partial x} \\
	\hat p_x \hat x \Psi = \hat p_x (\hat x \Psi) = - i\hbar \dfrac{\partial }{\partial x}(x\cdot \Psi) = -i\hbar \left(1\cdot\Psi + x \dfrac{\partial \Psi}{\partial x}\right)
\end{gather*}

В последнем равенстве в последней строчке мы это получили как производную произведения. Следовательно, $\{\hat x, \hat p_x\}\Psi = i\hbar\Psi \ne 0$. Т.е. мы получаем, что одновременно измерить координату и импльс невозможно (по одной оси), что как раз согласуется с соотношением неопределённостей Гейзенберга.

Также посмотриv, можно ли одновременно измерить координату по оси $O_x$ и проекцию импульса, но уже по оси $O_y$. Делаем всё аналогично:

\begin{gather*}
	\{\hat x, \hat p_y\} = \hat x \hat p_y - \hat p_y \hat x \\
	\hat x \hat p_y \Psi = \hat x (\hat p_y \Psi) = \hat x \cdot \left(- i\hbar \dfrac{\partial \Psi}{\partial y}\right) = -i\hbar x \dfrac{\partial \Psi}{\partial y} \\
	\hat p_y \hat x \Psi = \hat p_y (\hat x \Psi) = - i\hbar \dfrac{\partial }{\partial y}(x\cdot \Psi) = -i\hbar \left(\dfrac{\partial x}{\partial y} \cdot \Psi + x \dfrac{\partial \Psi}{\partial y}\right) = -i\hbar x \dfrac{\partial \Psi}{\partial y}
\end{gather*}

Следовательно, $\hat x \hat p_y = \hat p_y \hat x \Rightarrow \{\hat x, \hat p_y\} = 0$. Значит одновременно измерить координату по оси $O_x$ и проекцию импульса, но уже по оси $O_y$ можно.

\mysubsection{Момент импульса}

Момент импульса частицы $\vec L$ в классической механике выглядит так:

\[\vec L = [\vec r \; \vec p]\]

Но такое определение в квантовой механике не имеет смысла, т.е. не существует состояния, в котором оба вектора имеют определённое значение. В квантовой механике этому векторному произведению соответствует оператор:

\begin{gather*}
	\hat{\vec L} = [\hat{\vec r}, \; \hat{\vec p}] \\
	\hat{\vec r} = \hat x \vec i + \hat y \vec j + \hat z \vec k \\
	\hat{\vec p} = -i\hbar \left(\dfrac{\partial}{\partial x} \vec i + \dfrac{\partial}{\partial y} \vec j + \dfrac{\partial}{\partial z} \vec k\right) \\
	\hat{\vec L} = \hat L_x \vec i + \hat L_y \vec j + \hat L_z \vec k \\
	\hat{\vec L} =
	\begin{vmatrix}
	\vec i & \vec j & \vec k \\
	\hat x & \hat y & \hat z \\
	\hat p_x & \hat p_y & \hat p_z
	\end{vmatrix}
\end{gather*}

Раскрывая векторное произведение, получаем:






















\end{document}